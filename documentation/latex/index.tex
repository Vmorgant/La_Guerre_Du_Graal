\hypertarget{index_intro_sec}{}\section{Présentation du jeu}\label{index_intro_sec}
Ce jeu a été réalisé dans le cadre du projet d’algorithmique de deuxième année de licence S\+PI. 2 équipes de composées de différents personnage s\textquotesingle{}affrontent dans un combat à mort. Chaque classe de personnage possède ses forces et faiblesses, ces classes s\textquotesingle{}inspirent du jeu et de l\textquotesingle{}animé Fate Stay Night.~\newline
 Vous devez utiliser un système unix pour pouvoir utiliser ce programme.\hypertarget{index_install_sec}{}\section{Installation}\label{index_install_sec}
\hypertarget{index_step1}{}\subsection{Etape 1 \+: Téléchargement}\label{index_step1}
Télécharger les sources du jeu \href{https://github.com/Vmorgant/Projet_Algo.git}{\tt https\+://github.\+com/\+Vmorgant/\+Projet\+\_\+\+Algo.\+git}~\newline
~\newline
 Télécharger les sources de la S\+DL 2.\+0 (nécessaire pour la version 2.\+0) \href{http://libsdl.org/download-2.0.php}{\tt http\+://libsdl.\+org/download-\/2.\+0.\+php}~\newline
~\newline
 Télécharger les sources de la S\+D\+L\+\_\+ttf2 (nécessaire pour la version 2.\+0) \href{https://www.libsdl.org/projects/SDL_ttf/}{\tt https\+://www.\+libsdl.\+org/projects/\+S\+D\+L\+\_\+ttf/}~\newline
~\newline
 Ouvrir un terminal(pour les étapes suivantes).~\newline
~\newline
\hypertarget{index_step2}{}\subsection{Etape 2 \+: Installation de la S\+D\+L 2.\+0 et sdl 2.\+ttf (seulement pour la version 2.\+0)}\label{index_step2}
Extraire les archives obtenues lors du téléchargement.~\newline
~\newline
 Entrez la commande \char`\"{}cd\char`\"{} pour vous placer à la racine puis \char`\"{}mkdir S\+D\+L\char`\"{} pour créer le dossier de la S\+DL.~\newline
~\newline
 Déplacez vous dans le dossier où vous avez extrait l\textquotesingle{}archive de la S\+D\+L2.~\newline
~\newline
 Entrez la commande \+:~\newline
 -\/\char`\"{}./configure \char`\"{} si vous êtes super-\/utilisateur ~\newline
 -\/\char`\"{}./configure -\/-\/prefix=\$\+H\+O\+M\+E/\+S\+D\+L\char`\"{} sinon~\newline
~\newline
 Entrez la commande \char`\"{}make\char`\"{}~\newline
~\newline
 Entrez la commande \+:~\newline
 -\/\char`\"{}sudo make install\char`\"{} si vous êtes super-\/utilisateur ~\newline
 -\/\char`\"{}make install\char`\"{} sinon~\newline
~\newline
 Déplacez vous dans le dossier où vous avez extrait l\textquotesingle{}archive de la S\+D\+L\+\_\+ttf.~\newline
~\newline
 Entrez la commande \+:~\newline
 -\/\char`\"{}./configure \char`\"{} si vous êtes super-\/utilisateur -\/\char`\"{}./configure -\/-\/prefix=\$\+H\+O\+M\+E/\+S\+D\+L\char`\"{} sinon Entrez la commande \char`\"{}make\char`\"{}~\newline
~\newline
 Entrez la commande \+:~\newline
 -\/\char`\"{}sudo make install\char`\"{} si vous êtes super-\/utilisateur ~\newline
 -\/\char`\"{}make install\char`\"{} sinon~\newline
~\newline
\hypertarget{index_step3}{}\subsection{Etape 3 \+: Installation de la S\+D\+L 2.\+0 et sdl 2.\+ttf (seulement pour la version 2.\+0)}\label{index_step3}
Extraire les archives obtenues lors du téléchargement.~\newline
~\newline
 Ouvrir un terminal.~\newline
~\newline
 Entrez la commande \char`\"{}cd\char`\"{} pour vous placer à la racine puis \char`\"{}mkdir S\+D\+L\char`\"{} pour créer le dossier de la S\+DL.~\newline
~\newline
 Déplacez vous dans le dossier où vous avez extrait l\textquotesingle{}archive de la S\+D\+L2.~\newline
~\newline
 Entrez la commande \+:~\newline
 -\/\char`\"{}./configure \char`\"{} si vous êtes super-\/utilisateur~\newline
 -\/\char`\"{}./configure -\/-\/prefix=\$\+H\+O\+M\+E/\+S\+D\+L\char`\"{} sinon~\newline
~\newline
 Entrez la commande \char`\"{}make\char`\"{}~\newline
~\newline
 Entrez la commande \+:~\newline
 -\/\char`\"{}sudo make install\char`\"{} si vous êtes super-\/utilisateur ~\newline
 -\/\char`\"{}make install\char`\"{} sinon~\newline
~\newline
 Déplacez vous dans le dossier où vous avez extrait l\textquotesingle{}archive de la S\+D\+L\+\_\+ttf.~\newline
~\newline
 Entrez la commande \+:~\newline
 -\/\char`\"{}./configure \char`\"{} si vous êtes super-\/utilisateur ~\newline
 -\/\char`\"{}./configure -\/-\/prefix=\$\+H\+O\+M\+E/\+S\+D\+L\char`\"{} sinon~\newline
~\newline
 Entrez la commande \char`\"{}make\char`\"{}~\newline
~\newline
 Entrez la commande \+:~\newline
 -\/\char`\"{}sudo make install\char`\"{} si vous êtes super-\/utilisateur ~\newline
 -\/\char`\"{}make install\char`\"{} sinon~\newline
~\newline
\hypertarget{index_step4}{}\subsection{Etape 4 \+: Installation du jeu}\label{index_step4}
Déplacez vous dans le dossier où vous avez extrait l\textquotesingle{}archive du jeu.~\newline
~\newline
 Utilisez la commande make.~\newline
~\newline
 Lancez l\textquotesingle{}exécutable L\+G\+D\+G.\+out~\newline
~\newline
